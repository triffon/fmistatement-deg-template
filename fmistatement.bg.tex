% да се смени phd с dsc, ако становището е за доктор на науките
\documentclass[bulgarian,phd]{fmirsdeg}

\thesistitle{\emph{заглавие на дисертацията}}

\area{\emph{научна област}}

\field{\emph{професионално направление}}

\programme{\emph{докторска програма}}

\chair{\emph{катедра}}

\reviewer{\emph{академична длъжност, научна степен, име, презиме, фамилия - месторабота на рецензента}}{\emph{професионално направление на рецензента}}

\order{\emph{номер на заповедта}}{\emph{дата на заповедта}}

\candidatename{\emph{академична длъжност, научна степен, име, презиме, фамилия}}

\pages{\emph{брой страници}}

\chapters{\emph{брой глави}}

\refs{\emph{брой източници}}

\lang{\emph{български или английски}}

\begin{statement}

  % да се смени positive с negative, ако заключението е отрицателно
  \positiveconclusion
  % вътре в текста името на кандидата може да се реферира с \candidate или \Candidate (с първа главна буква)

  \papers{1}
  % могат да се добавят и други видове работи
  % \studia{\emph{брой}}
  % \monographs{\emph{брой}}
  % \books{\emph{брой}}
  % \patents{\emph{брой}}
  % \textbooks{\emph{брой}}

  % да се разкоментира при нужда
  % \otherdocs{\emph{брой на другите документи, ако са представени такива (във вид на служебни бележки и удостоверения от работодател, ръководител на проект, финансираща организация или възложител на проект, референции и отзиви, награди и други подходящи доказателства)}}{\emph{Бележки и коментар по документите.}}

  \begin{generalthesiscomment}
    \emph{Общи коментари по дисертационния труд.}
  \end{generalthesiscomment}

  \begin{generaldocscomment}
    \emph{Общи коментари по предадените документи.}
  \end{generaldocscomment}

  \begin{candidatedata}
    \emph{Кратки професионални и биографични данни за кандидата.}
  \end{candidatedata}

  \begin{resultsreview}
    \emph{Кратко описание къде са представяни съответните резултати и на базата на кои публикации е оформен представения дисертационен труд. Какво е отражението на резултатите на кандидата в трудовете на други автори. Числови показатели – цитати, импакт-фактор и др. При колективни публикации да се отрази приносът на кандидата.}
  \end{resultsreview}

  \begin{scientificreview}
    \emph{Анализ и оценка на научните и научно-приложните приноси на кандидата, съдържащи се в представения дисертационен труд, оригиналността и приложението им в други научни  постижения и в практиката.}
  \end{scientificreview}

  \begin{autorefreview}
    \emph{Посочват се информация за автореферата като се изказва становище дали отговаря / не отговаря на всички изисквания за изготвянето му, както и дали представя коректно резултатите и съдържанието на дисертационния труд.}
  \end{autorefreview}

  \begin{recommendations}
    \emph{Посочват се критични бележки, отправени към рецензираните трудове по отношение на: постановка; анализи и обобщения; методично равнище; точност и пълнота на резултатите; литературна осведоменост.}
  \end{recommendations}

  \begin{personalcomments}
    \emph{Лични впечатления от кандидата}
  \end{personalcomments}
\end{statement}
